\documentclass{article}
\usepackage{amsmath}
\usepackage{amssymb}
\usepackage{mathrsfs}
\usepackage{ragged2e}
\usepackage[utf8]{inputenc}
\usepackage{hyperref}
\DeclareMathOperator{\Q}{\mathbb{Q}}
\DeclareMathOperator{\R}{\mathbb{R}}
\DeclareMathOperator{\E}{\mathbb{E}}
\DeclareMathOperator{\F}{\mathcal{F}}
\DeclareMathOperator{\Prop}{\mathbb{P}}
\DeclareMathOperator{\1}{1\!\!1}
\DeclareMathOperator{\Bias}{Bias}
\DeclareMathOperator{\MSE}{MSE}
\DeclareMathOperator{\qed}{QED}
\begin{document}
\section{Gravity}
Newton Gravity: 
\begin{flalign*}
	\mathbf{\ddot x}=-G{m\over r^2}\mathbf{\hat x}&&
\end{flalign*}
\section{Circle Orbit}
A circular Orbit has following property.
\begin{flalign*}
	\mathbf{x}=&\,r e^{i \omega t}&&\\
	\mathbf{\dot x}=&\,i\,\omega r e^{i \omega t}\\
	\mathbf{\ddot x}=&-\omega^2 r e^{i \omega t}\\
	\\
	\Rightarrow&\\
	\\
	\omega =& \sqrt{\ddot x\over r}\\
	\\
	\Rightarrow&\\
	\\
	\mathbf{\dot x} =& \, \sqrt{\ddot x\over r}\,i\mathbf{x}
\end{flalign*}
\section{Correction Position}
For the correction of position a non-linear Gauss-Seidel solver can be used. \url {http://matthias-mueller-fischer.ch/realtimephysics/coursenotes.pdf}
\begin{flalign*}
	\Delta \mathbf p_i = - sw_i\nabla_{\mathbf p_i}C(\mathbf p)\\
	s={C(\mathbf p)\over\sum_j w_j |\nabla _{\mathbf p_j}C(\mathbf p)|^2}
\end{flalign*}
\end{document}